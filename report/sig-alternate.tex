% This is "sig-alternate.tex" V2.0 May 2012
% This file should be compiled with V2.5 of "sig-alternate.cls" May 2012
%
% This example file demonstrates the use of the 'sig-alternate.cls'
% V2.5 LaTeX2e document class file. It is for those submitting
% articles to ACM Conference Proceedings WHO DO NOT WISH TO
% STRICTLY ADHERE TO THE SIGS (PUBS-BOARD-ENDORSED) STYLE.
% The 'sig-alternate.cls' file will produce a similar-looking,
% albeit, 'tighter' paper resulting in, invariably, fewer pages.
%
% ----------------------------------------------------------------------------------------------------------------
% This .tex file (and associated .cls V2.5) produces:
%       1) The Permission Statement
%       2) The Conference (location) Info information
%       3) The Copyright Line with ACM data
%       4) NO page numbers
%
% as against the acm_proc_article-sp.cls file which
% DOES NOT produce 1) thru' 3) above.
%
% Using 'sig-alternate.cls' you have control, however, from within
% the source .tex file, over both the CopyrightYear
% (defaulted to 200X) and the ACM Copyright Data
% (defaulted to X-XXXXX-XX-X/XX/XX).
% e.g.
% \CopyrightYear{2007} will cause 2007 to appear in the copyright line.
% \crdata{0-12345-67-8/90/12} will cause 0-12345-67-8/90/12 to appear in the copyright line.
%
% ---------------------------------------------------------------------------------------------------------------
% This .tex source is an example which *does* use
% the .bib file (from which the .bbl file % is produced).
% REMEMBER HOWEVER: After having produced the .bbl file,
% and prior to final submission, you *NEED* to 'insert'
% your .bbl file into your source .tex file so as to provide
% ONE 'self-contained' source file.
%
% ================= IF YOU HAVE QUESTIONS =======================
% Questions regarding the SIGS styles, SIGS policies and
% procedures, Conferences etc. should be sent to
% Adrienne Griscti (griscti@acm.org)
%
% Technical questions _only_ to
% Gerald Murray (murray@hq.acm.org)
% ===============================================================
%
% For tracking purposes - this is V2.0 - May 2012

\documentclass{sig-alternate}

\pagenumbering{arabic}

\begin{document}
%
% --- Author Metadata here ---
\conferenceinfo{CS846}{'15 Waterloo, Canada}
\CopyrightYear{2015} % Allows default copyright year (20XX) to be over-ridden - IF NEED BE.
%\crdata{0-12345-67-8/90/01}  % Allows default copyright data (0-89791-88-6/97/05) to be over-ridden - IF NEED BE.
% --- End of Author Metadata ---

\title{Drawing Diagrams in Source Code
\titlenote{(Produces the permission block, and copyright information). 
For use with SIG-ALTERNATE.CLS. Supported by ACM.}} 

\numberofauthors{2}

\author{
\alignauthor
Reza Adhitya Saputra\\
       \affaddr{University of Waterloo}\\
       \email{radhitya@uwaterloo.ca}
% 2nd. author
\alignauthor
Raminder Sodhi\\
       \affaddr{University of Waterloo}\\
       \email{rjsodhi@uwaterloo.ca}
}


\maketitle
\begin{abstract}
An image tells a thousand words. Developers often use diagrams to explain their source code. Unfortunately, diagrams, which usually are drawn on papers or whiteboards, are transitory. However, we believe that embedding diagrams in source code can be used for improving code documentation. In this paper, we propose a tool for Github to embed diagrams in source code files as Unicode Art.
\end{abstract}

%% A category with the (minimum) three required fields
%\category{H.4}{Information Systems Applications}{Miscellaneous}
%%A category including the fourth, optional field follows...
%\category{D.2.8}{Software Engineering}{Metrics}[complexity measures, performance measures]
%
%\terms{Theory}

\keywords{Diaggramming, visual representation}

\section{Introduction}
Diagrams and Images are useful to explain algorithms or complex source code. Developers often draw simple diagrams as source code comments using ASCII characters directly in their text editors. However, drawing ASCII art is tedious and the result is simple. We propose a web-based tool that consist of a single canvas where programmers can draw a diagram then automatically converted to a source code comment as Unicode art.

\section{Diagramming Practice in OSS}
Why do developers draw diagrams?


\section{Diagramming Tool}

\section{User Study}

\section{Related Work}
\cite{Yatani2009} and \cite{Eunyoung2010} is related to drawing diagram in OSS.



\section{Conclusions}
This paper is done

%ACKNOWLEDGMENTS are optional

%\section{Acknowledgments}
%Thanks for all the fish

\nocite{*}
\bibliographystyle{abbrv}
\bibliography{reference_846}  % sigproc.bib is the name of the Bibliography in this case

\end{document}
